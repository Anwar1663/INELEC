\documentclass{article}
\usepackage{graphicx, float}
\usepackage{circuitikz}

\title{CircuiTikz Cheatsheet}
\date{Last Edit: \today}

\begin{document}
\maketitle

    \section{Options}

    \subsection{General}
        \begin{center}
            change color: \textit{color=red/blue/green...} \\
            change fill color: \textit{fill=red/blue/green...} \\
            change size: \textit{scale=2}\\
            Mirror: \textit{mirror}\\
            Invert: \textit{invert}\\
            line thickness: \textit{thickness=2}
        \end{center}
    \subsection{Node-style Components}
    They follow this syntax: 
    
    \#1: Component name.
    
    \#2: List of options seperated by a comma.
    
    \#3: Name of an anchor.
    
    \#4: Text written to the text anchor of the component.
    
    \#5: Curly Braces (Mandatory, even if empty).
        \begin{center}
            mirroring: \textit{xscale=-1}\\
            flipping: \textit{yscale=-1}
        \end{center}

        


    %--------------------------------------------------------------------------------------
    % MonoPoles
    %--------------------------------------------------------------------------------------

    \newpage
    \section{Monopoles}
    They Have the Type of node, Thus they use the syntax:\\
    \centering
        \textit{node[name]\{label\}} \\
    \raggedright
    \subsection{ground}
    \centering \begin{circuitikz}
        \draw (0,0) node[ground]{};
    \end{circuitikz} 
    
    \raggedright
    \subsection{tlground(Tail-less Ground)}
    \centering \begin{circuitikz}
        \draw (0,0) node[tlground]{};
    \end{circuitikz} 
    
    \raggedright
    \subsection{rground(Reference Ground)}
    \centering \begin{circuitikz}
        \draw (0,0) node[rground]{};
    \end{circuitikz} 
    
    \raggedright
    \subsection{pground(Protective Ground)}
    \centering \begin{circuitikz}
        \draw (0,0) node[pground]{};
    \end{circuitikz} 
    
    \raggedright
    \subsection{cground(Chassis Ground)}
    \centering \begin{circuitikz}
        \draw (0,0) node[cground]{};
    \end{circuitikz} 
    
    \raggedright
    \subsection{eground(European Ground)}
    \centering \begin{circuitikz}
        \draw (0,0) node[eground]{};
    \end{circuitikz} 
    
    \raggedright
    \subsection{eground2(European Ground 2)}
    \centering \begin{circuitikz}
        \draw (0,0) node[eground2]{};
    \end{circuitikz} 
    
    \raggedright
    
    
    %--------------------------------------------------------------------------------------
    % Bipoles
    %--------------------------------------------------------------------------------------
    
    
    \newpage
    \section{BiPoles}
    They have this syntax:\\
    \begin{center}
         \textit{(x1,y1) to [name] (x2, y2)}
    \end{center} 
    \raggedright There are other options that can be added inside the "[" and "]",
    you can pass more than one option, but you have to separate the options with a colon ","\\
    \textbf{Example:} [name of component, options 1, option 2, option 3] \\
    \centering 
    Label: [l = name of label] \\
    Label (on the other side): [l\_= Name of Label] \\
    Voltage Sign: [v = Name of Voltage] \\
    Voltage Sign (Inverted Polarity): [v<=Name of Voltage]\\
    Changing Polarity: [invert] \\
    Arrow: [->] OR [<-] OR [<->] \\
    Open Circuit(Circles): [-o] OR [o-]

    
    
    \raggedright
    \subsection{R(american resistor)}
    \begin{center}
        \begin{circuitikz}
            \draw (0,0) to [resistor] (2,0);
        \end{circuitikz}
    \end{center}
    
    \subsection{vR(Variable resistor)}
    \begin{center}
        \begin{circuitikz}
            \draw (0,0) to [resistor] (2,0);
        \end{circuitikz}
    \end{center}
    
    \subsection{european resistor}
    \begin{center}
        \begin{circuitikz}
            \draw (0,0) to [european resistor] (2,0);
        \end{circuitikz}
    \end{center}
    
    \subsection{V (DC Voltage Source)}
    \begin{center}
        \begin{circuitikz}[american]
            \draw (0,0) to [V, invert] (0,1.5);
        \end{circuitikz}
    \end{center}
    
    \subsection{cV (Controlled Voltage Source)}
    \begin{center}
        \begin{circuitikz}[american]
            \draw (0,0) to [cV, invert] (0,1.5);
        \end{circuitikz}
    \end{center}
    
    \subsection{sV (AC Voltage Source)}
    \begin{center}
        \begin{circuitikz}[american]
            \draw (0,0) to [sV, invert] (0,1.5);
        \end{circuitikz}
    \end{center}
    
    \subsection{csV (Controlled Alternating Voltage Source)}
    \begin{center}
        \begin{circuitikz}[american]
            \draw (0,0) to [csV, invert] (0,1.5);
        \end{circuitikz}
    \end{center}
    
    \subsection{battery}
    \begin{center}
        \begin{circuitikz}[american]
            \draw (0,0) to [battery] (2,0);
        \end{circuitikz}
    \end{center}
    
    \subsection{battery1(Single Cell Battery)}
    \begin{center}
        \begin{circuitikz}[american]
            \draw (0,0) to [battery1] (2,0);
        \end{circuitikz}
    \end{center}
    
    \subsection{I (DC Source)}
    \begin{center}
        \begin{circuitikz}[american]
            \draw (0,0) to [I] (0,1.5);
        \end{circuitikz}
    \end{center}
    
    \subsection{cI (Controlled Current Source)}
    \begin{center}
        \begin{circuitikz}[american]
            \draw (0,0) to [cI] (0,1.5);
        \end{circuitikz}
    \end{center}
    
    \subsection{sI (AC Source)}
    \begin{center}
        \begin{circuitikz}[american]
            \draw (0,0) to [sI, invert] (0,1.5);
        \end{circuitikz}
    \end{center}
    
    \subsection{csI (Controlled Alternating Current Source)}
    \begin{center}
        \begin{circuitikz}[american]
            \draw (0,0) to [csI, invert] (0,1.5);
        \end{circuitikz}
    \end{center}
    
    \subsection{C (Capacitor)}
    \begin{center}
        \begin{circuitikz}[american]
            \draw (0,0) to [C] (2,0);
        \end{circuitikz}
    \end{center}
    
    \subsection{cC (Polarized Capacitor)}
    \begin{center}
        \begin{circuitikz}[american]
            \draw (0,0) to [cC] (2,0);
        \end{circuitikz}
    \end{center}
    
    \subsection{L (Inductor)}
    \begin{center}
        \begin{circuitikz}[american]
            \draw (0,0) to [L] (2,0);
        \end{circuitikz}
    \end{center}
    
    \subsection{vL (Variable Inductor)}
    \begin{center}
        \begin{circuitikz}[american]
            \draw (0,0) to [vL] (2,0);
        \end{circuitikz}
    \end{center}
    
    \subsection{voltmeter}
    \begin{center}
        \begin{circuitikz}[american]
            \draw (0,0) to [voltmeter] (2,0);
        \end{circuitikz}
    \end{center}
    
    \subsection{ammeter}
    \begin{center}
        \begin{circuitikz}[american]
            \draw (0,0) to [ammeter] (2,0);
        \end{circuitikz}
    \end{center}
    
    \subsection{ohmmeter}
    \begin{center}
        \begin{circuitikz}[american]
            \draw (0,0) to [ohmmeter] (2,0);
        \end{circuitikz}
    \end{center}
    
    \subsection{nos (normal open switch)}
    \begin{center}
        \begin{circuitikz}[american]
            \draw (0,0) to [nos] (2,0);
        \end{circuitikz}
    \end{center}
    
    \subsection{ncs (normal closed switch)}
    \begin{center}
        \begin{circuitikz}[american]
            \draw (0,0) to [ncs] (2,0);
        \end{circuitikz}
    \end{center}
    
    \subsection{switch(closing switch)}
    \begin{center}
        \begin{circuitikz}[american]
            \draw (0,0) to [switch] (2,0);
        \end{circuitikz}
    \end{center}
    
    \subsection{fuse}
    \begin{center}
        \begin{circuitikz}[american]
            \draw (0,0) to [fuse] (2,0);
        \end{circuitikz}
    \end{center}
    
    \subsection{afuse(asymmetric)}
    \begin{center}
        \begin{circuitikz}[american]
            \draw (0,0) to [afuse] (2,0);
        \end{circuitikz}
    \end{center}
    
    \subsection{thermocouple}
    \begin{center}
        \begin{circuitikz}[american]
            \draw (0,0) to [thermocouple] (2,0);
        \end{circuitikz}
    \end{center}
    
    \subsection{lamp}
    \begin{center}
        \begin{circuitikz}[american]
            \draw (0,0) to [lamp] (2,0);
        \end{circuitikz}
    \end{center}
    
    \subsection{bulb}
    \begin{center}
        \begin{circuitikz}[american]
            \draw (0,0) to [bulb] (2,0);
        \end{circuitikz}
    \end{center}
    
    
    
    
    %--------------------------------------------------------------------------------------
    % DIODES
    %--------------------------------------------------------------------------------------
   
   
   
    \newpage
    \section{Diodes}
    Empty Diodes End With: Do (o is empty)\\
    Full Diodes End With: D* (full)\\
    Stroke Diodes End with: D- \\
    If omitted The Default is an empty Diode

    \begin{center}
    \Huge Comparison: \normalsize
        \begin{circuitikz}[american]
            \draw (-4,0) to [D, l= Empty] (-2,0)
            (-1,0) to [D*, l=Full](1,0)
            (2, 0) to [D-, l=Stroke](4,0);
        \end{circuitikz}
    \end{center}
    
    
    
    \subsection{Do (Empty Diode)}
    \begin{center}
        \begin{circuitikz}[american]
            \draw (0,0) to [Do] (2,0);
        \end{circuitikz}
    \end{center}
    
    \subsection{zDo (Empty Zener Diode)}
    \begin{center}
        \begin{circuitikz}[american]
            \draw (0,0) to [zDo] (2,0);
        \end{circuitikz}
    \end{center}
    
    \subsection{zzDo (Empty ZZener Diode)}
    \begin{center}
        \begin{circuitikz}[american]
            \draw (0,0) to [zzDo] (2,0);
        \end{circuitikz}
    \end{center}
    
    \subsection{leDo (Empty Led)}
    \begin{center}
        \begin{circuitikz}[american]
            \draw (0,0) to [leDo] (2,0);
        \end{circuitikz}
    \end{center}
    
    \subsection{pDo (Empty PhotoDiode)}
    \begin{center}
        \begin{circuitikz}[american]
            \draw (0,0) to [pDo] (2,0);
        \end{circuitikz}
    \end{center}
    
    \subsection{sDo (Empty Schottky Diode)}
    \begin{center}
        \begin{circuitikz}[american]
            \draw (0,0) to [sDo] (2,0);
        \end{circuitikz}
    \end{center}
    
    \subsection{biDo (Empty Biderectional Diode)}
    \begin{center}
        \begin{circuitikz}[american]
            \draw (0,0) to [biDo] (2,0);
        \end{circuitikz}
    \end{center}
    
    \subsection{tvsDo (Empty TVS Diode)}
    \begin{center}
        \begin{circuitikz}[american]
            \draw (0,0) to [tvsDo] (2,0);
        \end{circuitikz}
    \end{center}
    
     \subsection{Tr (Triac)}
    \begin{center}
        \begin{circuitikz}[american]
            \draw (0,0) to [Tr] (2,0);
        \end{circuitikz}
    \end{center}
    
    \subsection{Ty (Thyristor)}
    \begin{center}
        \begin{circuitikz}[american]
            \draw (0,0) to [Ty] (2,0);
        \end{circuitikz}
    \end{center}
    
    \subsection{PUT (PUT)}
    \begin{center}
        \begin{circuitikz}[american]
            \draw (0,0) to [PUT] (2,0);
        \end{circuitikz}
    \end{center}
    
    
    %--------------------------------------------------------------------------------------
    % Transistors
    %--------------------------------------------------------------------------------------
    
    
    \newpage
    \section{Transistors}
    Transistors have type: node. \\
    They follow this syntax:\\
    \begin{center}
        \textit{(x1,y1) node[name, options]\{label\}}
    \end{center}
    Options are: 
    \begin{center}
        nobase: remove the base
    \end{center}
    every N transistor has its P image

    \subsection{npn(pnp)}
    \begin{center}
        \begin{circuitikz}[american]
            \draw (0,0) node[npn]{};
        \end{circuitikz}
    \end{center}
    
    \subsection{npn(pnp), bodydiode}
    \begin{center}
        \begin{circuitikz}[american]
            \draw (0,0) node[npn, bodydiode]{};
        \end{circuitikz}
    \end{center}
    
    \subsection{npn(pnp), schottky base}
    \begin{center}
        \begin{circuitikz}[american]
            \draw (0,0) node[npn, schottky base]{};
        \end{circuitikz}
    \end{center}
    
    \subsection{npn(pnp), photo}
    \begin{center}
        \begin{circuitikz}[american]
            \draw (0,0) node[npn, photo]{};
        \end{circuitikz}
    \end{center}
    
    \subsection{bjtnpn(pnp), collectors=$n$, emitters=$m$}
    \begin{center}
        \begin{circuitikz}[american]
            \draw (0,0) node[bjtnpn, emitters=1 ,collectors=2]{};
        \end{circuitikz}
    \end{center}
    
    \subsection{n(p)mos}
    \begin{center}
        \begin{circuitikz}[american]
            \draw (0,0) node[nmos]{};
        \end{circuitikz}
    \end{center}

    
    \subsection{n(p)mosd (Depletion)}
    \begin{center}
        \begin{circuitikz}[american]
            \draw (0,0) node[nmosd]{};
        \end{circuitikz}
    \end{center}

    
    \subsection{hemt}
    \begin{center}
        \begin{circuitikz}[american]
            \draw (0,0) node[hemt]{};
        \end{circuitikz}
    \end{center}
    
    
    %--------------------------------------------------------------------------------------
    % Amplifiers
    %--------------------------------------------------------------------------------------
    
    
    \newpage
    \section{Amplifiers}
    Transistors have type: node. \\
    They follow this syntax:\\
    \begin{center}
        \textit{(x1,y1) node[name, options](name of amplifier)\{label\}}\\
        Connect to a specific port: \textit{(name of amplifier.port)}
    \end{center}
    
     Names of ports are: 
    \begin{center}
        positive post: \textit{name.+}\\
        negative port: \textit{name.-}\\
        Output: \textit{name.out}\\
        up: \textit{name.up}\\
        down: \textit{name.down} \\
        \textbf{for Fully differential op amps:}\\
        Positive output: \textit{name.out +}\\
        negative output: \textit{name.out -}\\[0.3cm]
        
        \begin{circuitikz}[american]
            \draw (0,0) node[op amp](op1){}
            (op1.-) node[left]{$negative$}
            (op1.+) node[left]{$positive$}
            (op1.up) to ++(0,0.5) node[above]{$up$}
            (op1.down) to ++(0,-0.5) node[below]{$down$}
            (op1.out) node[right]{$out$};
        \end{circuitikz}
    \end{center}
    
    Options are: 
    \begin{center}
        input polarity: \textit{noinv input up(down)}\\
        output polarity: \textit{noinv output up(down)}
    \end{center}

    \subsection{op amp}
    \begin{center}
        \begin{circuitikz}[american]
            \draw (0,0) node[op amp](op1){};
        \end{circuitikz}
    \end{center}
    
    
    \subsection{fd op amp(fully differential operational amplifier)}
    \begin{center}
        \begin{circuitikz}[american]
            \draw (0,0) node[fd op amp](op2){};
        \end{circuitikz}
    \end{center}
    
    
    %--------------------------------------------------------------------------------------
    % Logic Gates
    %--------------------------------------------------------------------------------------
    
    
    \newpage
    \section{IEEE Logic Gates}
    Transistors have type: node. \\
    They follow this syntax:\\
    \begin{center}
        \textit{(x1,y1) node[name, options](name of gate)\{label\}}\\
        Connect to a specific port: \textit{(name of gate.port)}
    \end{center}
    
    \begin{center}
        input 1: \textit{name.in 1}\\
        input 2: \textit{name.in 2}\\
        Output: \textit{name.out}\\[0.3cm]
        
        \begin{circuitikz}[american]
            \draw (0,0) node[ieeestd and port](gate){}
            (gate.in 1) node[left]{input 1}
            (gate.in 2) node[left]{input 2}
            (gate.out) node[right]{output};
        \end{circuitikz}
    \end{center}
    
    \subsection{ieeestd and port(IEEE AND PORT)}
    \begin{center}
        \begin{circuitikz}[american]
            \draw (0,0) node[ieeestd and port](gate){};
        \end{circuitikz}
    \end{center}
    
    \subsection{ieeestd nand port(IEEE NAND PORT)}
    \begin{center}
        \begin{circuitikz}[american]
            \draw (0,0) node[ieeestd nand port](gate){};
        \end{circuitikz}
    \end{center}
    
    \subsection{ieeestd or port(IEEE OR PORT)}
    \begin{center}
        \begin{circuitikz}[american]
            \draw (0,0) node[ieeestd or port](gate){};
        \end{circuitikz}
    \end{center}
    
    \subsection{ieeestd nor port(IEEE NOR PORT)}
    \begin{center}
        \begin{circuitikz}[american]
            \draw (0,0) node[ieeestd nor port](gate){};
        \end{circuitikz}
    \end{center}
    
    \subsection{ieeestd xor port(IEEE XOR PORT)}
    \begin{center}
        \begin{circuitikz}[american]
            \draw (0,0) node[ieeestd xor port](gate){};
        \end{circuitikz}
    \end{center}
    
    \subsection{ieeestd xnor port(IEEE XNOR PORT)}
    \begin{center}
        \begin{circuitikz}[american]
            \draw (0,0) node[ieeestd xnor port](gate){};
        \end{circuitikz}
    \end{center}
    
    \subsection{ieeestd buffer port(IEEE BUFFER PORT)}
    \begin{center}
        \begin{circuitikz}[american]
            \draw (0,0) node[ieeestd buffer port](gate){};
        \end{circuitikz}
    \end{center}
    
    \subsection{ieeestd not port(IEEE NOT PORT)}
    \begin{center}
        \begin{circuitikz}[american]
            \draw (0,0) node[ieeestd not port](gate){};
        \end{circuitikz}
    \end{center}
    
    \subsection{ieeestd schmitt port(IEEE SCHMITT PORT)}
    \begin{center}
        \begin{circuitikz}[american]
            \draw (0,0) node[ieeestd schmitt port](gate){};
        \end{circuitikz}
    \end{center}
    
    
    %--------------------------------------------------------------------------------------
    % Logic Gates
    %--------------------------------------------------------------------------------------
    
    
    \newpage
    \section{Flip-flops}
    Transistors have type: node. \\
    They follow this syntax:\\
    \begin{center}
        \textit{(x1,y1) node[name, options](name of flipflop)\{label\}}\\
        Connect to a specific port: \textit{(name of flipflop.port)}
    \end{center}
    
    \begin{center}
        pin 1: \textit{name.bpin 1}\\
        pin 2: \textit{name.bpin 2}\\
        pin 3: \textit{name.bpin 3}\\
        pin 4: \textit{name.bpin 4}\\
        pin 5: \textit{name.bpin 5}\\
        pin 6: \textit{name.bpin 6}\\
        up: \textit{name.bup}\\
        down: \textit{name.bdown}
    \end{center}
    Other Options are:
    \begin{center}
        set pin width \textit{external pins width=1} \\
        negate a pin: \textit{<node name>-N<pin number>} ex: J-N5\\
        
    \end{center}
    
    \subsection{flipflop (empty)}
    \begin{center}
        \begin{circuitikz}[american]
            \draw (0,0) node[flipflop](ff){};
        \end{circuitikz}
    \end{center}
    
    \subsection{flipflop}
    \begin{center}
        \begin{circuitikz}[american]
            \draw (0,0) node[flipflop](ff){}
            (ff.bpin 1) to ++(-0.3,0) node[left]{pin 1} 
            (ff.bpin 2) to ++(-0.3,0) node[left]{pin 2} 
            (ff.bpin 3) to ++(-0.3,0) node[left]{pin 3} 
            (ff.bpin 4) to ++(0.3,0) node[right]{pin 4} 
            (ff.bpin 5) to ++(0.3,0) node[right]{pin 5}
            (ff.bpin 6) to ++(0.3,0) node[right]{pin 6}
            (ff.bup) to ++(0,+0.3) node[above]{up}
            (ff.bdown) to ++(0,-0.3) node[below]{down};
        \end{circuitikz}
    \end{center}
    
    \subsection{latch}
    \begin{center}
        \begin{circuitikz}[american]
            \draw (0,0) node[latch](latch){}
            (latch.pin 1) node[left]{name.pin 1}
            (latch.pin 3) node[left]{name.pin 3}
            (latch.pin 4) node[right]{name.pin 4}
            (latch.pin 6) node[right]{name.pin 6};
        \end{circuitikz}
    \end{center}
    
    \subsection{flipflop SR}
    \begin{center}
        \begin{circuitikz}[american]
            \draw (0,0) node[flipflop SR](sr){}
            (sr.pin 1) node[left]{name.pin 1}
            (sr.pin 3) node[left]{name.pin 3}
            (sr.pin 4) node[right]{name.pin 4}
            (sr.pin 6) node[right]{name.pin 6};
        \end{circuitikz}
    \end{center}
    
    \subsection{flipflop JK}
    \begin{center}
        \begin{circuitikz}[american]
            \draw (0,0) node[flipflop JK](jk){}
            (jk.pin 1) node[left]{name.pin 1}
            (jk.pin 2) node[left]{name.pin 2}
            (jk.pin 3) node[left]{name.pin 3}
            (jk.pin 4) node[right]{name.pin 4}
            (jk.pin 6) node[right]{name.pin 6};
        \end{circuitikz}
    \end{center}
    
    \subsection{flipflop JK, add async SR (synchronous flip-flop JK)}
    \begin{center}
        \begin{circuitikz}[american]
            \draw (0,0) node[flipflop JK, add async SR](jk){}
            (jk.pin 1) node[left]{name.pin 1}
            (jk.pin 2) node[left]{name.pin 2}
            (jk.pin 3) node[left]{name.pin 3}
            (jk.pin 4) node[right]{name.pin 4}
            (jk.pin 6) node[right]{name.pin 6}
            (jk.up) node[above]{name.up}
            (jk.down) node[below]{name.down};
        \end{circuitikz}
    \end{center}
    
    
\end{document}