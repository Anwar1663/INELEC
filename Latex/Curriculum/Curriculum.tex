\documentclass[12pt, a4paper]{article}

% These are the packages used in this file:
\usepackage{outlines}
\usepackage{enumitem}
\usepackage{hyperref}
\usepackage{setspace}

% This is the the setup for \href.
% the available options & types are:
% hyperindex bool, linktocpage bool, breaklinks bool
% colorlinks bool, linkcolor string (red, blue, magenta...)
% anchorcolor string, citecolor string, filecolor string
% urlcolor string, frenchlinks bool
\hypersetup{
    colorlinks=true,
    urlcolor=blue}
\setenumerate[1]{label=I}

\title{\Huge L2 Curriculum}
\date{Last Edit: \today}

\begin{document}
    \maketitle
    \newpage
    \begin{outline}
     \Huge 
       \1 First Semester \Large 
         \2 Active Devices I
         \2 Active Devices I - LAB
         \2 Ordinary Differential Equations
         \2 Digital Systems I with VHDL
         \2 Digital Systems I - LAB
         \2 Electrical Engineering II
         \2 Electrical Engineering II - LAB
         \2 Physics III 
         \2 Physics III - LAB
         \newpage 
     \Huge 
       \1 \Huge Second Semester \Large 
         \2 Active Devices II 
         \2 Active Devices II - LAB
         \2 Linear Systems
         \2 Information and Communications Techniques
         \2 Electric Machines
         \2 Digital Systems II with VHDL
         \2 Digital Systems II - LAB
         \2 ElectroMagnetic Field Theory
         \2 Probability and Statistics
    \end{outline}
    
\newpage
\large This file contains only the titles of each chapter, to find a more details 
about each chapter and what it countains please refer back to the original Drive where
this file was first uploaded by 
\href{https://www.youtube.com/watch?v=dQw4w9WgXcQ}{clicking here}

\newpage
\vspace*{\fill}
\begin{center}
    \Huge Semester I %(Physics Lab Unfinished)
\end{center}
\vspace*{\fill}
 
\newpage 
\begin{center}
\Huge 
Active Devices I \\ (EE241)
\end{center}
\normalsize
\large \underline{- Chapter 1:}  Introduction To SemiConductor Materials\\ \\
\large \underline{- Chapter 2:}  SemiConductor Dioedes And Their Applications\\ \\ 
\large \underline{- Chapter 3:}  Bipolar junction transistors (BJT’s)\\ \\ 
\large \underline{- Chapter 4:}  BJT Amplifiers small signal low frequency analysis and design\\ \\



\newpage 
\begin{center}
\Huge 
Active Devices I - LAB  \\ (EE241-L)
\end{center}
\normalsize
\large \underline{- LAB 1:} Diodes \\ \\
\large \underline{- LAB 2:} Half-Wave  and Full-Wave Rectification \\ \\
\large \underline{- LAB 3:} Clipper \& Clamper Circuits\\ \\
\large \underline{- LAB 4:} Zener Diods \\ \\
\large \underline{- LAB 5:} LED Characterization \\ \\
\large \underline{- LAB 6:} Basic BJT Characteristics \\ \\


\newpage 
\begin{center}
\Huge 
Ordinary Differential Equations \\ (EE271)
\end{center}
\normalsize
\large \underline{- Chapter 1:} EigenValues And EigenVectors\\ \\
\large \underline{- Chapter 2:} First And Second Order Differential Equations\\ \\ 
\large \underline{- Chapter 3:} Laplace Transformation\\ \\ 
\large \underline{- Chapter 4:} System of Linear Differential Equations\\ \\ 


\newpage 
\begin{center}
\Huge 
Digital Systems I with VHDL\\ (EE221)
\end{center}
\normalsize
\large \underline{- Chapter 1:} Number Systems \& Data Representation\\ \\
\large \underline{- Chapter 2:} Two-Valued Boolean Algebra\\ \\
\large \underline{- Chapter 3:} Minimizing Techniques\\ \\
\large \underline{- Chapter 4:} Introduction to VHDL\\ \\
\large \underline{- Chapter 5:} Combinational Circuits (at Gate Level)\\ \\
\large \underline{- Chapter 6:} Combinational Circuits using VHDL\\ \\
\large \underline{- Chapter 7:} Integrated Circuit Logic families\\ \\


\newpage 
\begin{center}
\Huge 
Digital Systems I with VHDL - LAB  \\ (EE221-L) %unfinished 
\end{center}

\begin{center} \normalsize These Labs are unfinished \end{center}
\large \underline{- LAB 5:} Logic circuit design and Implementation using TTL ICs \\ \\
\large \underline{- LAB 6:} Introduction to Quartus II \& DE2 Board\\ \\
\large \underline{- LAB 7:} Introduction to VHDL modeling using dataflow design style \\ \\
\large \underline{- LAB 9:} Combinational circuit analysis and design with fixed-function ICs and VHDL\\ \\


\newpage 
\begin{center}
\Huge 
Electrical Engineering II \\ (EE203)
\end{center}
\normalsize
\large \underline{- Chapter 1:} AC Fundamentals and Sinusoidal Alternating Waveform\\ \\
\large \underline{- Chapter 2:} Transient\\ \\
\large \underline{- Chapter 3:} R, L and C Elements and Impedance Concept\\ \\
\large \underline{- Chapter 4:} Steady State Analysis\\ \\
\large \underline{- Chapter 5:} AC Network Theorems\\ \\
\large \underline{- Chapter 6:} Power in AC Circuits\\ \\
\large \underline{- Chapter 7:} Frequency Response\\ \\
\large \underline{- Chapter 8:} Polyphase System \\ \\


\newpage 
\begin{center}
\Huge 
Electrical Engineering II - LAB (EE203-L)
\end{center}
\normalsize
\large \underline{- LAB 1:} Familiarization  \\ \\
\large \underline{- LAB 2:} Transients \\ \\
\large \underline{- LAB 3:} Reactances  \\ \\
\large \underline{- LAB 4:} Series Circuits \\ \\
\large \underline{- LAB 5:} Parallel Circuits  \\ \\
\large \underline{- LAB 6:} AC Thevenin Theorem \& Maximum Power Transfer \\ \\
\large \underline{- LAB 7:} Nodal \& Mesh Analysis  \\ \\
\large \underline{- LAB 8:} Frequency Response \\ \\
\large \underline{- LAB 9:} A Frequency Response of RLC Circuits And Resonance  \\ \\


\newpage 
\begin{center}
\Huge 
Physics III \\ (EE273)
\end{center}
\normalsize
\large \underline{- Chapter 1:} Simple Harmonic Motion\\ \\
\large \underline{- Chapter 2:} Damped Oscillations\\ \\
\large \underline{- Chapter 3:} Forced Oscillations\\ \\
\large \underline{- Chapter 4:} Coupled Oscillations


\newpage 
\begin{center}
\Huge 
Physics III - LAB \\(EE273-L) %Needs The name of the Labs
\end{center}
\normalsize
\large \underline{- LAB 1:} Free oscillations of systems with one degree of freedom  \\ \\
\large \underline{- LAB 2:} Forced oscillations of systems with one degree of freedom \\ \\

 \newpage
\vspace*{\fill}
\begin{center}
    \Huge Semester II %(Unfinished)
\end{center}
\vspace*{\fill}

    
\newpage 
\begin{center}
\Huge 
Active Devices II \\ (EE242)
\end{center}
\normalsize
\large \underline{- Chapter 1:} Power Amplifiers \\ \\
\large \underline{- Chapter 2:} Field-Effect Transistor (FET's)\\ \\ 
\large \underline{- Chapter 3:} FET Amplifiers Small Signal Low Frequency Analysis and Design \\ \\ 
\large \underline{- Chapter 4:} Differntial Amplifiers\\ \\
\large \underline{- Chapter 5:} Operational Amplifiers (Op-amps) and theirs Applications \\ \\
\large \underline{- Chapter 6}  Silicon Controlled Rectifiers (SCR) and other Devices




\newpage 
\begin{center}
\Huge 
Active Devices II - LAB  \\ (EE242-L) 
\end{center}
\normalsize
These Labs Start From 7 Since The \textquotedblleft Active Devices I" Had 6 Labs \\ \\ 
\large \underline{- LAB 7:} BJT AC Analysis (Common-Emitter configuration) \\ \\
\large \underline{- LAB 8:} BJT AC Analysis (Common-Base Configuration) \\ \\
\large \underline{- LAB 9:} Class B(AB) Push-Pull Power amplifier \\ \\
\large \underline{- LAB 10:} JFET Characteristic \\ \\
\large \underline{- LAB 11:} E-MOSFET Characteristics \\ \\
\large \underline{- LAB 12:} E-MOSFET: Switching mode \\ \\
\large \underline{- LAB 13:} JFET Amplifiers \\ \\
\large \underline{- LAB 14:} Differential amplifier \\ \\
\large \underline{- LAB 15:} Operational Ampliefiers (Op-Amps) \\ \\


\newpage 
\begin{center}
\Huge 
Linear Systems \\ (EE252) 
\end{center}
\normalsize
\large \underline{- Chapter 1:} Time Signal and Systems\\ \\
\large \underline{- Chapter 2:} Time-Invariant Systems\\ \\ 
\large \underline{- Chapter 3:} Continuous-Time Fourier Series\\ \\ 
\large \underline{- Chapter 4:} Continuous-Time Fourier Transform\\ \\ 
\large \underline{- Chapter 5:} The Laplace Transform


\newpage 
\begin{center}
\Huge 
Information and Communication Techniques \\ (EE281)
\end{center}
\normalsize
\large \underline{- Chapter 1:} Effective Communication\\ \\
\large \underline{- Chapter 2:} Writing Engineering Documents\\ \\
\large \underline{- Chapter 3:} Lab Reports\\ \\
\large \underline{- Chapter 4:} Correspondence\\ \\
\large \underline{- Chapter 5:} Resources\\ \\


\newpage 
\begin{center}
\Huge 
Electric Machines \\ (EE232)
\end{center}
\normalsize
\large \underline{- Chapter 1:} Single \& Three-Phase Circuits\\ \\
\large \underline{- Chapter 2:} Magnetic Circuits\\ \\
\large \underline{- Chapter 3:} Transformers\\ \\
\large \underline{- Chapter 4:} DC Machines\\ \\
\large \underline{- Chapter 5:} Synchronous Machines\\ \\
\large \underline{- Chapter 6:} Asynchronous Machines\\ \\
\large \underline{- Chapter 7:} Special Machines\\ \\



\newpage 
\begin{center}
\Huge 
Electric Machines - LAB  \\ (EE232-L) 
\end{center}
\normalsize
\large \underline{- LAB 1:} Safety and The Power Supply \\ \\
\large \underline{- LAB 2:} The Wattmeter \\ \\
\large \underline{- LAB 3:} Three Phase Circuits \\ \\
\large \underline{- LAB 4:} Single Phase Transformer\\ \\
\large \underline{- LAB 5:} DC Motor \\ \\



\newpage 
\begin{center}
\Huge 
Digital Systems II with VHDL\\ (EE222)\\ 
\end{center}
\normalsize
\large \underline{- Chapter 1:} Sequential Circuits\\ \\
\large \underline{- Chapter 2:} Structural Design Style\\ \\
\large \underline{- Chapter 3:} Standard Combinational Modules Decoders\\ \\


\newpage 
\begin{center}
\Huge 
Digital Systems II with VHDL - LAB  \\ (EE222-L) %unfinished 
\end{center}
\large \underline{- LAB 1:} Design, Simulation and Implementation of
Flip-Flops Using ICs and VHDL \\ \\
\large \underline{- LAB 2:} Design and Implementation of Digital Counters Using ICs and Schematic Capture\\ \\
\large \underline{- LAB 3:} Design and Implementation of Digital Counters Using MSIs and VHDL \\ \\
\large \underline{- LAB 4:} Applications of Digital Counters Using VHDL\\ \\
\large \underline{- LAB 5:} Registers / Shift Register and Applications \\ \\
\large \underline{- LAB 6:} Standard Combinational Modules and Applications \\ \\



\newpage 
\begin{center}
\Huge 
ElectroMagnetic Field Theory \\ (EE262)
\end{center}
\large
\underline{- Chapter 1:} Vector overview and coordinate systems\\ \\
\underline{- Chapter 2:} \ \ - Line, surface, and volume integral \\
\indent \indent \indent \indent \indent - Gradient, Divergence, and Curl  operators \\
\indent \indent \indent \indent \indent - Divergence Theorem \\
\indent \indent \indent \indent \indent - Stockes's theorem  \\ \\
\underline{- Chapter 3:} Electrostatic and Magnetostatic field theory \\ \\
\underline{- Chapter 4:} Maxwell's Equations \\ \\
\underline{- Chapter 5:} Uniform Plane wave: \\
\indent \indent \indent \indent \indent - Propagation in free space  \\ 
\indent \indent \indent \indent \indent - Propagation in Dielectrics \\
\indent \indent \indent \indent \indent - Propagation in conductive material 


\newpage 
\begin{center}
\Huge 
Probability And Statistics \\ (EE246)
\end{center}
\normalsize
\large \underline{- Chapter 1:} Probability\\ \\
\large \underline{- Chapter 2:} Concept of Random Variable\\ \\
\large \underline{- Chapter 3:} Multiple Random Variables\\ \\
\large \underline{- Chapter 4:} Estimation Theory\\ \\
\large \underline{- Chapter 5:} Tests of Hypothesis\\ \\
\large \underline{- Chapter 6:} Linear Regression\\ \\
    
\end{document}

