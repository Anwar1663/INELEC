\documentclass{article}
\usepackage{graphicx, float}
\usepackage{circuitikz}

\title{CircuiTikz Summary}
\date{Last Edit: \today}

\begin{document}
\maketitle

    %--------------------------------------------------------------------------------------
    % MonoPoles
    %--------------------------------------------------------------------------------------


    \section{Monopoles}
    They Have the Type of node, Thus they use the syntax:\\
    \centering
        \textit{node[name]\{label\}} \\
    \raggedright
    \subsection{ground}
    \centering \begin{circuitikz}
        \draw (0,0) node[ground]{};
    \end{circuitikz} 
    
    \raggedright
    \subsection{tlground(Tail-less Ground)}
    \centering \begin{circuitikz}
        \draw (0,0) node[tlground]{};
    \end{circuitikz} 
    
    \raggedright
    \subsection{rground(Reference Ground)}
    \centering \begin{circuitikz}
        \draw (0,0) node[rground]{};
    \end{circuitikz} 
    
    \raggedright
    \subsection{pground(Protective Ground)}
    \centering \begin{circuitikz}
        \draw (0,0) node[pground]{};
    \end{circuitikz} 
    
    \raggedright
    \subsection{cground(Chassis Ground)}
    \centering \begin{circuitikz}
        \draw (0,0) node[cground]{};
    \end{circuitikz} 
    
    \raggedright
    \subsection{eground(European Ground)}
    \centering \begin{circuitikz}
        \draw (0,0) node[eground]{};
    \end{circuitikz} 
    
    \raggedright
    \subsection{eground2(European Ground 2)}
    \centering \begin{circuitikz}
        \draw (0,0) node[eground2]{};
    \end{circuitikz} 
    
    \raggedright
    
    
    %--------------------------------------------------------------------------------------
    % Bipoles
    %--------------------------------------------------------------------------------------
    
    
    \newpage
    \section{BiPoles}
    They have this syntax:\\
    \begin{center}
         \textit{(x1,y1) to [name] (x2, y2)}
    \end{center} 
    \raggedright There are other options that can be added inside the "[" and "]",
    you can pass more than one option, but you have to separate the options with a colon ","\\
    \textbf{Example:} [name of component, options 1, option 2, option 3] \\
    \centering 
    Label: [l = name of label] \\
    Label (on the other side): [l\_= Name of Label] \\
    Voltage Sign: [v = Name of Voltage] \\
    Voltage Sign (Inverted Polarity): [v<=Name of Voltage]\\
    Changing Polarity: [invert] \\
    Arrow: [->] OR [<-] OR [<->] \\
    Open Circuit(Circles): [-o] OR [o-]

    
    
    \raggedright
    \subsection{R(american resistor)}
    \begin{center}
        \begin{circuitikz}
            \draw (0,0) to [resistor] (2,0);
        \end{circuitikz}
    \end{center}
    
    \subsection{vR(Variable resistor)}
    \begin{center}
        \begin{circuitikz}
            \draw (0,0) to [resistor] (2,0);
        \end{circuitikz}
    \end{center}
    
    \subsection{european resistor}
    \begin{center}
        \begin{circuitikz}
            \draw (0,0) to [european resistor] (2,0);
        \end{circuitikz}
    \end{center}
    
    \subsection{V (DC Voltage Source)}
    \begin{center}
        \begin{circuitikz}[american]
            \draw (0,0) to [V, invert] (0,1.5);
        \end{circuitikz}
    \end{center}
    
    \subsection{cV (Controlled Voltage Source)}
    \begin{center}
        \begin{circuitikz}[american]
            \draw (0,0) to [cV, invert] (0,1.5);
        \end{circuitikz}
    \end{center}
    
    \subsection{sV (AC Voltage Source)}
    \begin{center}
        \begin{circuitikz}[american]
            \draw (0,0) to [sV, invert] (0,1.5);
        \end{circuitikz}
    \end{center}
    
    \subsection{csV (Controlled Alternating Voltage Source)}
    \begin{center}
        \begin{circuitikz}[american]
            \draw (0,0) to [csV, invert] (0,1.5);
        \end{circuitikz}
    \end{center}
    
    \subsection{battery}
    \begin{center}
        \begin{circuitikz}[american]
            \draw (0,0) to [battery] (2,0);
        \end{circuitikz}
    \end{center}
    
    \subsection{battery1(Single Cell Battery)}
    \begin{center}
        \begin{circuitikz}[american]
            \draw (0,0) to [battery1] (2,0);
        \end{circuitikz}
    \end{center}
    
    \subsection{I (DC Source)}
    \begin{center}
        \begin{circuitikz}[american]
            \draw (0,0) to [I] (0,1.5);
        \end{circuitikz}
    \end{center}
    
    \subsection{cI (Controlled Current Source)}
    \begin{center}
        \begin{circuitikz}[american]
            \draw (0,0) to [cI] (0,1.5);
        \end{circuitikz}
    \end{center}
    
    \subsection{sI (AC Source)}
    \begin{center}
        \begin{circuitikz}[american]
            \draw (0,0) to [sI, invert] (0,1.5);
        \end{circuitikz}
    \end{center}
    
    \subsection{csI (Controlled Alternating Current Source)}
    \begin{center}
        \begin{circuitikz}[american]
            \draw (0,0) to [csI, invert] (0,1.5);
        \end{circuitikz}
    \end{center}
    
    \subsection{C (Capacitor)}
    \begin{center}
        \begin{circuitikz}[american]
            \draw (0,0) to [C] (2,0);
        \end{circuitikz}
    \end{center}
    
    \subsection{cC (Polarized Capacitor)}
    \begin{center}
        \begin{circuitikz}[american]
            \draw (0,0) to [cC] (2,0);
        \end{circuitikz}
    \end{center}
    
    \subsection{L (Inductor)}
    \begin{center}
        \begin{circuitikz}[american]
            \draw (0,0) to [L] (2,0);
        \end{circuitikz}
    \end{center}
    
    \subsection{vL (Variable Inductor)}
    \begin{center}
        \begin{circuitikz}[american]
            \draw (0,0) to [vL] (2,0);
        \end{circuitikz}
    \end{center}
    
    \subsection{voltmeter}
    \begin{center}
        \begin{circuitikz}[american]
            \draw (0,0) to [voltmeter] (2,0);
        \end{circuitikz}
    \end{center}
    
    \subsection{ammeter}
    \begin{center}
        \begin{circuitikz}[american]
            \draw (0,0) to [ammeter] (2,0);
        \end{circuitikz}
    \end{center}
    
    \subsection{ohmmeter}
    \begin{center}
        \begin{circuitikz}[american]
            \draw (0,0) to [ohmmeter] (2,0);
        \end{circuitikz}
    \end{center}
    
    \subsection{fuse}
    \begin{center}
        \begin{circuitikz}[american]
            \draw (0,0) to [fuse] (2,0);
        \end{circuitikz}
    \end{center}
    
    \subsection{afuse(asymmetric)}
    \begin{center}
        \begin{circuitikz}[american]
            \draw (0,0) to [afuse] (2,0);
        \end{circuitikz}
    \end{center}
    
    \subsection{thermocouple}
    \begin{center}
        \begin{circuitikz}[american]
            \draw (0,0) to [thermocouple] (2,0);
        \end{circuitikz}
    \end{center}
    
    \subsection{lamp}
    \begin{center}
        \begin{circuitikz}[american]
            \draw (0,0) to [lamp] (2,0);
        \end{circuitikz}
    \end{center}
    
    \subsection{bulb}
    \begin{center}
        \begin{circuitikz}[american]
            \draw (0,0) to [bulb] (2,0);
        \end{circuitikz}
    \end{center}
    
    
    
    %--------------------------------------------------------------------------------------
    % DIODES
    %--------------------------------------------------------------------------------------
   
   
   
    \newpage
    \section{Diodes}
    Empty Diodes End With: Do (o is empty)\\
    Full Diodes End With: D* (full)\\
    Stroke Diodes End with: D- \\
    If omitted The Default is an empty Diode

    \begin{center}
    \Huge Comparison: \normalsize
        \begin{circuitikz}[american]
            \draw (-4,0) to [D, l= Empty] (-2,0)
            (-1,0) to [D*, l=Full](1,0)
            (2, 0) to [D-, l=Stroke](4,0);
        \end{circuitikz}
    \end{center}
    
    
    
    \subsection{Do (Empty Diode)}
    \begin{center}
        \begin{circuitikz}[american]
            \draw (0,0) to [Do] (2,0);
        \end{circuitikz}
    \end{center}
    
    \subsection{zDo (Empty Zener Diode)}
    \begin{center}
        \begin{circuitikz}[american]
            \draw (0,0) to [zDo] (2,0);
        \end{circuitikz}
    \end{center}
    
    \subsection{zzDo (Empty ZZener Diode)}
    \begin{center}
        \begin{circuitikz}[american]
            \draw (0,0) to [zzDo] (2,0);
        \end{circuitikz}
    \end{center}
    
    \subsection{leDo (Empty Led)}
    \begin{center}
        \begin{circuitikz}[american]
            \draw (0,0) to [leDo] (2,0);
        \end{circuitikz}
    \end{center}
    
    \subsection{pDo (Empty PhotoDiode)}
    \begin{center}
        \begin{circuitikz}[american]
            \draw (0,0) to [pDo] (2,0);
        \end{circuitikz}
    \end{center}
    
    \subsection{sDo (Empty Schottky Diode)}
    \begin{center}
        \begin{circuitikz}[american]
            \draw (0,0) to [sDo] (2,0);
        \end{circuitikz}
    \end{center}
    
    \subsection{biDo (Empty Biderectional Diode)}
    \begin{center}
        \begin{circuitikz}[american]
            \draw (0,0) to [biDo] (2,0);
        \end{circuitikz}
    \end{center}
    
    \subsection{tvsDo (Empty TVS Diode)}
    \begin{center}
        \begin{circuitikz}[american]
            \draw (0,0) to [tvsDo] (2,0);
        \end{circuitikz}
    \end{center}
    
     \subsection{Tr (Triac)}
    \begin{center}
        \begin{circuitikz}[american]
            \draw (0,0) to [Tr] (2,0);
        \end{circuitikz}
    \end{center}
    
    \subsection{Ty (Thyristor)}
    \begin{center}
        \begin{circuitikz}[american]
            \draw (0,0) to [Ty] (2,0);
        \end{circuitikz}
    \end{center}
    
    \subsection{PUT (PUT)}
    \begin{center}
        \begin{circuitikz}[american]
            \draw (0,0) to [PUT] (2,0);
        \end{circuitikz}
    \end{center}
    
    \newpage
    \section{Transistors}
    Transistors have type: node. \\
    They follow this syntax:\\
    \begin{center}
        \textit{(x1,y1) node[name]{label}}
    \end{center}
    
    
    
    
    
    
    
    
    
\end{document}